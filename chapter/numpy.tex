
\chapter{NumPy}

\begin{minted}[linenos,fontfamily=courier]{python}
  import numpy as np
\end{minted}

\begin{minted}[linenos,fontfamily=courier]{python}
  ary.ndim     
  ary.size
  ary.shape
  ary.shape[0]
  ary.shape[1]
\end{minted}


\section{様々な初期化メソッド}

\begin{minted}[linenos,fontfamily=courier]{python}
  >>> np.ones([3,3])
  array([[1., 1., 1.],
       [1., 1., 1.],
       [1., 1., 1.]])
\end{minted}


\begin{minted}[linenos,fontfamily=courier]{python}
  >>> np.zeros([3,3])
  array([[0., 0., 0.],
       [0., 0., 0.],
       [0., 0., 0.]])
\end{minted}

\begin{minted}[linenos,fontfamily=courier]{python}
  >>> np.eye(3)
  array([[1., 0., 0.],
       [0., 1., 0.],
       [0., 0., 1.]])
\end{minted}

\begin{minted}[linenos,fontfamily=courier]{python}
  >>> np.diag([3,3,3])
  array([[3, 0, 0],
       [0, 3, 0],
       [0, 0, 3]])
\end{minted}

\begin{minted}[linenos,fontfamily=courier]{python}
  >>> np.arange(4,10)
  array([4, 5, 6, 7, 8, 9])

  >>> np.arange(5)
  array([0, 1, 2, 3, 4])

  >>> np.arange(1,10,1)
  array([1, 2, 3, 4, 5, 6, 7, 8, 9])

  >>> np.linspace(0,10,5)
  array([ 0. ,  2.5,  5. ,  7.5, 10. ])
\end{minted}


\section{要素同士の演算}
{\tt ndarray}の基本は多次元配列であり、各要素に対して演算させることができる。
基本の四則演算に加えて、{\tt ufunc}も各要素に対して使用することができる。

\begin{minted}[linenos,fontfamily=courier]{python}
>>> ary1 = np.array([[1,2,3],[4,5,6]])
>>> ary2 = np.array([[7,8,9],[10,11,12]])
>>> ary1 + ary2
array([[ 8, 10, 12],
       [14, 16, 18]])
\end{minted}
\begin{minted}[linenos,fontfamily=courier]{python}
>>> ary1 - ary2
array([[-6, -6, -6],
       [-6, -6, -6]])
\end{minted}

\begin{minted}[linenos,fontfamily=courier]{python}
>>> ary1 * ary2
array([[ 7, 16, 27],
       [40, 55, 72]])
\end{minted}

\begin{minted}[linenos,fontfamily=courier]{python}
>>> ary1 / ary2
array([[0.14285714, 0.25      , 0.33333333],
       [0.4       , 0.45454545, 0.5       ]])
\end{minted}

\begin{minted}[linenos,fontfamily=courier]{python}
>>> ary1**2
array([[ 1,  4,  9],
       [16, 25, 36]])
\end{minted}

\begin{minted}[linenos,fontfamily=courier]{python}
>>> np.log(ary1)
array([[0.        , 0.69314718, 1.09861229],
       [1.38629436, 1.60943791, 1.79175947]])
\end{minted}

\begin{minted}[linenos,fontfamily=courier]{python}
>>> np.sqrt(ary1)
array([[1.        , 1.41421356, 1.73205081],
       [2.        , 2.23606798, 2.44948974]])
\end{minted}

行列としての演算を行うためには、明示的に関数を使用する必要がある\footnote{
注意しなければならないのは、便利な{\tt broadcast}等の機能があるにせよ、
行列計算を行うためには、正しく数学上の規則にのっとる必要がある。
例えば上記の{\tt ary1}と{\tt ary2}をそのまま内積計算させると

\begin{minted}[linenos,fontfamily=courier]{python}
>>> np.dot(ary1, ary2)
Traceback (most recent call last):
  File "<stdin>", line 1, in <module>
  File "<__array_function__ internals>", line 6, in dot
ValueError: shapes (2,3) and (2,3) not aligned: 3 (dim 1) != 2 (dim 0)
\end{minted}

となり、$2\times 3$行列と$2\times 3$行列の積は定義できないとエラーがでる。}。

{\tt array}は各要素の和も取ることができる。
また{\tt axis}を指定することで、行方向への和({\tt axis=1})、列方向への和({\tt axis=0})も計算することができる。

\begin{minted}[linenos,fontfamily=courier]{python}
>>> ary1.sum()
21
>>> ary1.sum(axis=0)
array([5, 7, 9])
>>> ary1.sum(axis=1)
array([ 6, 15])
\end{minted}


\section{配列の情報}

\begin{minted}[linenos,fontfamily=courier]{python}
>>> ary1.ndim
2
>>> ary1.size
6
>>> ary1.shape
(2, 3)
>>> ary1.shape[0]
2
>>> ary1.shape[1]
3
\end{minted}

\section{{\tt ufuncs}まとめ}

\begin{minted}[linenos,fontfamily=courier]{python}
>>> ary1.std()
1.707825127659933
>>> ary1.var()
2.9166666666666665
>>> np.sort(ary1)
array([[1, 2, 3],
       [4, 5, 6]])
       
# 要素の中の最小値
>>> np.min(ary1)
1
# 要素の中の最小値のインデックス
>>> np.argmin(ary1)
0
# 要素の中の最大値
>>> np.max(ary1)
6
# 要素の中の最大値のインデックス
>>> np.argmax(ary1)
5
\end{minted}


